%==============================================================
% History
%==============================================================
\begin{table}[H]
   %\begin{adjustbox}{width=\textwidth, totalheight=\textheight}
    \begin{adjustbox}{width=\textwidth}
    \textsf{
    \begin{tabular}{|L{5mm}{}{t}|L{10mm}{}{t}|L{10mm}{}{t}|L{150mm}{130mm}{t}|L{10mm}{}{t}|}
        \hline
        \setMultiColumn{4}{180mm}{170mm}{l}{t}{|}{|}{\textbf{Revision History}} &
        \setMultiRow{2}{Note}
        \nextRow \cline{1-4}
        %----------------
        \textbf{Rev} &
        \textbf{Date} &
        \textbf{Author} &
        \textbf{Description} &
         ~
        \nextRow \hline
        %---- Rev.01 ----
        01 & Nov.19 2021 & MM &
1st Release. &
        ~
        \nextRow \hline
        %---- Rev.02 ----
            02 & Dec.31 2021 & MM &
(1) Added MRET and WFI descriptions.\lb 
(2) Supported Questa as logic simulator. To do so, add an option -voptargs="+acc" in vsim command.\lb
(3) Supported Initialization of Instruction RAM in FPGA using .mif file. A conversion tool hex2mif is added in tools directory.\lb
(4) Updated JTAG interface schematic.\lb
(5) Changed operation of application mmRISC\_SampleCPU.\lb
(6) Add a retro text video game StarTrek as an application. &
        ~
        \nextRow \hline
        %---- Rev.03 ----
        03 & Apr.30 2022 & MM &
(1) verilog/common/defines.v is divided into defines\_core.v for mmRISC Core and defines\_chip.v for Chip System. (defines.v is not used any more.)\lb
(2) Added I2C, SPI, and SDRAM interface in Top Layer.\lb
(3) Added know-hows to use Raspberry Pi as a Development Environment including OpenOCD interface.\lb
(4) In application mmRISC\_SampleCPU, added access of Acceleration Sensor on MAX10-Lite board through I2C interface.\lb
(5) Added new application mmRISC\_TouchLCD which handles Adafruit-2-8-tft-touch-shield-v2 with Resistive Touch Panel or Capacitive Touch Panel for Arduino. \lb
(6) In each sample program, baud rate of UART is unified to 115200bps.  &
        ~
        \nextRow \hline
        %---- Rev.04 ----
        04 & Nov.20 2023 & MM &
(1) Added cJTAG (2-wire compact JTAG) as the debug interface. You can select the debug interface from JTAG or cJTAG.\lb
(2) Added alternative Halt-On-Reset, controlled by a hardware input signal level, instead of the one in standard RISC-V debug function.\lb
(3) Added a new JTAG DR register for user's multi purposes, for example, to configure the operation modes of the SoC from JTAG/cJTAG interface.\lb
(4) The number of 32bit secure code for authentication is expanded from one to two.\lb
(5) Supported low power mode (STBY).\lb
(6) Added precise verification methods for floating point operations. Corrected RTL code in conversion from float to int (cpu\_fpu32.v).\lb
(7) Added a application program; Tic-Tac-Toe AI Game on Touch LCD panel.\lb
(8) Re-write this document by using TexShop.&
        ~
        \nextRow \hline
        %---- Rev.05 ----
        05 & Sep.01 2024 & MM &
(1) You can switch the debug interface between 4-wire JTAG or 2-wire cJTAG by setting the signal level of "enable\_cjtag" in chip\_top\_wrap.v. This signal is connected to GPIO2[6] corresponding to SW6 on DE10-Lite board. Therefore from this version onward, there are only two FPGA configurations; one is RV32IMAFC\_JTAGcJTAG, which unifies previous RV32IMAFC\_JTAG and RV32IMAFC\_cJTAG; the other is RV32IMAC\_JTAGcJTAG, which unifies previous RV32IMAC\_JTAG and RV32IMAFC\_cJTAG.\lb
(2) Even in the previous version, the detection of escape sequences in "cJTAG\_2\_JTAG.v" operated correctly for cJTAG waves generated by OpenOCD. However, the logic could not correctly detect some cJTAG escape waves generated by other generic debuggers, such as commercial RISC-V development IDEs. Additionally, in "debug\_dtm\_jtag.v", the IR in the JTAG TAP controller set to 0x1F caused the TDO output to be fixed high, which led to failures in JTAG chain tests when shifting long bit chains. These bugs have been fixed in this version, and now the mmRISC-1 can connect to generic commercial IDEs correctly.\lb
(3) Fixed a bug in the pipeline stall control for the C.FSWSP instruction (RV32FC). In the previous version, if the destination of the previous instruction was FRn, and the C.FSWSP stored the same FRn, the C.FSWSP did not stall correctly.\lb
(4) In the sample application programs, the interrupt handler routines have been re-written using "\_\_attribute\_\_ ((interrupt))", and the control of interrupt nesting has been moved to the C routine "interrupt.c" from the assembler routine "startup.S".&
        ~
        \nextRow \hline
        %---- Rev.06 ----
        06 & Jan.26 2025 & MM &
(1) The address expression for the TINFO Register (CSR) of the Debug Trigger Module has been corrected from 0x7a3 to 0x7a4 in the document. This error was only present in the document and did not affect the RTL logic.\lb
(2) The cause field in the DCSR Register (CSR) of the Debug related function was previously set to 3’b011 (halt request) even when the cause was a Hardware Break Point. This has been corrected in the RTL description to 3’b010 (breakpoint exception by hardware trigger module) when the cause is a Hardware Break Point. The corrected RTL file is cpu\_pipeline.v. This cause field is used by some commercial IDEs to implement specific I/O emulation, such as printf().&
        ~
        \nextRow \hline
        %---- Rev.07 ----
        07 & Feb.01 2025 & MM &
(1) In the section of "Connection to the target FPGA as 2-wire cJTAG interface", added a figure to show the connection between the cJTAG debugger probe and the cJTAG signals.&
        ~
        \nextRow \hline
    \end{tabular}
    }
    \end{adjustbox}
\end{table}

\null\vfill
\centerline{\textbf{Copyright(c) 2023-2024 by Munetomo Maruyama}}
\centerline{\textbf{X (Twitter) : @Processing\_Unit}}


